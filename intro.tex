\section{Introducci\'on}
Tanto problemas de estad\'istica como de inteligencia artificial es posible encontrar dificultades al momento de determinar que informaci\'on es \'util o relevante para poder extraer conclusiones m\'as precisas respecto del fen\'omeno que se est\'e estudiando. Esta informaci\'on relevante o \'util no es simple de detectar al observar los datos debido a que estos podr\'ian tener una gran dimensionalidad o su influencia en los resultados no es del todo obvia. Para estos casos, en los cuales se busca construir un modelo de clasificac\'on sencillo pero a su vez lo m\'as calibrado posible se debe explorar la informaci\'on entregada a este y determinar que informaci\'on es la m\'as apropiada para tal acometido. Para este fin, es necesario hacer una selecci\'on de la informaci\'on la que se traduce en la selecci\'on de caracter\'isticas o atributos de las instancias de las clases que se quiere detectar. 
\bigskip

En este trabajo se pretende realizar un acercamiento al proceso de selecci\'on de atributos usando dos diferentes tipos de bases de datos: una relacionada con la determinaci\'on de diabetes y la otra relacionada con la determinaci\'on de d\'igitos a partir de informaci\'on obtenida de manuscritos. Este acercamiento se realizar\'a usando el software WEKA. 

\subsection{Estructura del informe}
Las siguientes secciones corresponden a: \textbf{Marco te\'orico}, en el cual se responde a preguntas planteadas respecto del la selecci\'on de caracter\'isticas, de su proceso y que tipo m\'etodos podr\'ian apoyar en el proceso. Posteriormente, en la secci\'on \ref{sec3} se estudian los resultados obtenidos para las diferentes experiencias con el conjunto de datos de diabetes. En la secci\'on \ref{sec4} se describe la transformaci\'on del archivo CSV de los datos de d\'igitos a un archivo ARFF (para poder cargarlo en el software WEKA), y de la clasificaci\'on sin y con selecci\'on de atributos para estos datos. Luego en la secci\'on \ref{sec5} se tiene un an\'alsis global de los resultados m\'as importantes, que luego son resumidos en la secci\'on de conclusiones \ref{sec6}.  