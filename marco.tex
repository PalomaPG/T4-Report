\section{Marco te\'orico}\label{sec2}
\subsection{Selecci\'on de caracter\'isticas}
En aprendizaje de m\'aquinas y estad\'istica, la \textit{selecci\'on de caracter\'isticas} corresponde esencialmente a la selecci\'on de un subconjunto de atributos relevantes para la construcci\'on de un modelo. Esta selecci\'on se realiza por las siguientes razones:
\begin{enumerate}
\item Simplificaci\'on del modelo para lograr una interpretaci\'on m\'as f\'acil de conseguir.
\item Disminuir los tiempos de entrenamiento
\item Para evitar la \textbf{maldici\'on de la dimensionalidad}~\cite{bellman}, la cual no s\'olo complejiza el entrenamiento, sino tambi\'en exige una mayor cantidad de datos de entrenamiento.
\item Lograr una mejor generalizaci\'on del modelo, reduciendo el overfitting (i.e. para reducir la varianza).
\end{enumerate}

Su finalidad es la de deshacerse de informaci\'on irrelevante o que no aporta informaci\'on significativa dentro del modelo. 
\subsection{Etapas de la selecci\'on de caracter\'isticas}
\begin{enumerate}
\item \textbf{Generaci\'on de subconjuntos:} Es un procedimiento de sondeo que genera subconjuntos candidatos de caracter\'isticas para la evaluaci\'on, bas\'andose en una \textit{estrategia de b\'usqueda}~\cite{liuyu}. 
\item \textbf{Evaluaci\'on:} Cada subconjunto candidato es evaluado y comparado con los resultados obtenidos por el mejor subconjunto previo (el que ha obtenido mejores resultados en clasificaci\'on) de acuerdo a un \textit{criterio de evaluaci\'on}. La generaci\'on de estos conjuntos depender\'a de la m\'etrica de evaluaci\'on, la cual puede corresponder a una de las siguientes categor\'ias: de filtrado, wrappers o m\'etodo embebido.
\item \textbf{Criterio de parada:}
\item \textbf{Validaci\'on:} El subconjunto seleccionado usualmente requiere de ser validado a trav\'es de conocimiento que se tenga de los datos o a trav\'es de diferentes pruebas usando datos sint\'eticos o reales.
\end{enumerate}
\subsection{Metodolog\'ias para la generaci\'on de subconjuntos}
\begin{enumerate}
\item \textbf{Completa:} 
\item \textbf{Heur\'istica:}
\item \textbf{Aleatoria:}
\end{enumerate}
\subsection{Metodolog\'ias de evaluaci\'on de subconjuntos}
